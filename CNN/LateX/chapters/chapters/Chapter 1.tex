\section{Code description}
Στην εργασία αυτήν χρησιμοποιήθηκε η βιβλιοθήκη tensorflow για την κατασκευή του RNN δικτύου. Το dataset που χρησιμοποιήθηκε ήταν tweets. Ο αριθμός των tweets έφτανε τα 1.600.000 και ήταν πανω σε sentimental analysis, συγκεκριμένα αν είναι θετικό ή αρνητικό το tweet(binary classification).
Αρχικά δημιουργήθηκε μία συνάρτηση η οποία φορτώνει το dataset, ανακατεύει τα tweets(αποφεύγουμε έτσι το overfitting), στην συνέχεια διαγράφουμε κάποιες στηλες που δεν χρειάζονται και επειδή έχουμε βάλει data limit κρατάμε τον αριθμό tweets που θα εκπαιδεύσουμε το νευρωνικό.
Στην συγκεκριμένα εργασία λόγω χρόνου και έλλειψης υπολογιστικής ισχύς κρατήθηκαν τα 100.000 tweets. Επίσης στην αρχή το νευρωνικό έτρεξε σε όλα τα tweets και δεν φάνηκε κάποια διαφορά με τα υπόλοιπα στο μικρότερο dataset. Μετά το data limitation, χρησιμοποιούμαι την βιβλιοθήκη regex ώστε να ορίσουμε τα url και τα username ώστε αργότερα να τα αφαιρέσουμε, διότι δεν μας δίνουν κάποια χρησιμη πληροφορία και είναι θόρυβος για το νευρωνικό, ενώ ταυτόχρονα κατεβάζουμε βιβλιοθήκε της NLTK οι οποίες χρησιμοποιούνται σε μία νέα συνάρτηση την (processtweets), κατα την οποία γίνονται τα εξής:
